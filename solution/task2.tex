
\documentclass[a4paper,12pt]{article}

\usepackage{cmap}					% поиск в PDF
\usepackage[T2A]{fontenc}			% кодировка
\usepackage[utf8]{inputenc}			% кодировка исходного текста
\usepackage[english,russian]{babel}	% локализация и переносы

\begin{document} % Конец преамбулы, начало текста.

\section{Решение квадратного уравнения}

\textit{Задача:} решить уравнение $2x^2+5x-12=0$

\textit{Решение.} Это квадратное уравнение, общий вид которого: \[ax^2+bx+c=0\]
В нашем случае $a=2$, $b=5$, $c=-12$.

Сначала необходимо вычислить дискриминант уравнения: \[D=b^2-4ac=(5)^2-4*2*(-12)=121\]
Так как дискриминант является положительным ($D>0$), это уравнение имеет два корня, вычисляемые по формуле:
\[x_{1,2}=\frac{-b\pm\sqrt{D}}{2a}=\frac{-5\pm11}{4}\]
Таким образом, $x_1=\frac{-16}{4}=-4$, $x_2=\frac{6}{4}=\frac{3}{2}$.

\textit{Ответ:} $x_1=-4$, $x_2=\frac{3}{2}$.
\end{document} % Конец текста.